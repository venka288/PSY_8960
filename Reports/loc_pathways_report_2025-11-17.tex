% Options for packages loaded elsewhere
\PassOptionsToPackage{unicode}{hyperref}
\PassOptionsToPackage{hyphens}{url}
\documentclass[
  english,
  man,floatsintext]{apa6}
\usepackage{xcolor}
\usepackage{amsmath,amssymb}
\setcounter{secnumdepth}{-\maxdimen} % remove section numbering
\usepackage{iftex}
\ifPDFTeX
  \usepackage[T1]{fontenc}
  \usepackage[utf8]{inputenc}
  \usepackage{textcomp} % provide euro and other symbols
\else % if luatex or xetex
  \usepackage{unicode-math} % this also loads fontspec
  \defaultfontfeatures{Scale=MatchLowercase}
  \defaultfontfeatures[\rmfamily]{Ligatures=TeX,Scale=1}
\fi
\usepackage{lmodern}
\ifPDFTeX\else
  % xetex/luatex font selection
\fi
% Use upquote if available, for straight quotes in verbatim environments
\IfFileExists{upquote.sty}{\usepackage{upquote}}{}
\IfFileExists{microtype.sty}{% use microtype if available
  \usepackage[]{microtype}
  \UseMicrotypeSet[protrusion]{basicmath} % disable protrusion for tt fonts
}{}
\makeatletter
\@ifundefined{KOMAClassName}{% if non-KOMA class
  \IfFileExists{parskip.sty}{%
    \usepackage{parskip}
  }{% else
    \setlength{\parindent}{0pt}
    \setlength{\parskip}{6pt plus 2pt minus 1pt}}
}{% if KOMA class
  \KOMAoptions{parskip=half}}
\makeatother
% Make \paragraph and \subparagraph free-standing
\makeatletter
\ifx\paragraph\undefined\else
  \let\oldparagraph\paragraph
  \renewcommand{\paragraph}{
    \@ifstar
      \xxxParagraphStar
      \xxxParagraphNoStar
  }
  \newcommand{\xxxParagraphStar}[1]{\oldparagraph*{#1}\mbox{}}
  \newcommand{\xxxParagraphNoStar}[1]{\oldparagraph{#1}\mbox{}}
\fi
\ifx\subparagraph\undefined\else
  \let\oldsubparagraph\subparagraph
  \renewcommand{\subparagraph}{
    \@ifstar
      \xxxSubParagraphStar
      \xxxSubParagraphNoStar
  }
  \newcommand{\xxxSubParagraphStar}[1]{\oldsubparagraph*{#1}\mbox{}}
  \newcommand{\xxxSubParagraphNoStar}[1]{\oldsubparagraph{#1}\mbox{}}
\fi
\makeatother
\usepackage{graphicx}
\makeatletter
\newsavebox\pandoc@box
\newcommand*\pandocbounded[1]{% scales image to fit in text height/width
  \sbox\pandoc@box{#1}%
  \Gscale@div\@tempa{\textheight}{\dimexpr\ht\pandoc@box+\dp\pandoc@box\relax}%
  \Gscale@div\@tempb{\linewidth}{\wd\pandoc@box}%
  \ifdim\@tempb\p@<\@tempa\p@\let\@tempa\@tempb\fi% select the smaller of both
  \ifdim\@tempa\p@<\p@\scalebox{\@tempa}{\usebox\pandoc@box}%
  \else\usebox{\pandoc@box}%
  \fi%
}
% Set default figure placement to htbp
\def\fps@figure{htbp}
\makeatother
% definitions for citeproc citations
\NewDocumentCommand\citeproctext{}{}
\NewDocumentCommand\citeproc{mm}{%
  \begingroup\def\citeproctext{#2}\cite{#1}\endgroup}
\makeatletter
 % allow citations to break across lines
 \let\@cite@ofmt\@firstofone
 % avoid brackets around text for \cite:
 \def\@biblabel#1{}
 \def\@cite#1#2{{#1\if@tempswa , #2\fi}}
\makeatother
\newlength{\cslhangindent}
\setlength{\cslhangindent}{1.5em}
\newlength{\csllabelwidth}
\setlength{\csllabelwidth}{3em}
\newenvironment{CSLReferences}[2] % #1 hanging-indent, #2 entry-spacing
 {\begin{list}{}{%
  \setlength{\itemindent}{0pt}
  \setlength{\leftmargin}{0pt}
  \setlength{\parsep}{0pt}
  % turn on hanging indent if param 1 is 1
  \ifodd #1
   \setlength{\leftmargin}{\cslhangindent}
   \setlength{\itemindent}{-1\cslhangindent}
  \fi
  % set entry spacing
  \setlength{\itemsep}{#2\baselineskip}}}
 {\end{list}}
\usepackage{calc}
\newcommand{\CSLBlock}[1]{\hfill\break\parbox[t]{\linewidth}{\strut\ignorespaces#1\strut}}
\newcommand{\CSLLeftMargin}[1]{\parbox[t]{\csllabelwidth}{\strut#1\strut}}
\newcommand{\CSLRightInline}[1]{\parbox[t]{\linewidth - \csllabelwidth}{\strut#1\strut}}
\newcommand{\CSLIndent}[1]{\hspace{\cslhangindent}#1}
\ifLuaTeX
\usepackage[bidi=basic]{babel}
\else
\usepackage[bidi=default]{babel}
\fi
% get rid of language-specific shorthands (see #6817):
\let\LanguageShortHands\languageshorthands
\def\languageshorthands#1{}
\ifLuaTeX
  \usepackage[english]{selnolig} % disable illegal ligatures
\fi
\setlength{\emergencystretch}{3em} % prevent overfull lines
\providecommand{\tightlist}{%
  \setlength{\itemsep}{0pt}\setlength{\parskip}{0pt}}
% Manuscript styling
\usepackage{upgreek}
\captionsetup{font=singlespacing,justification=justified}

% Table formatting
\usepackage{longtable}
\usepackage{lscape}
% \usepackage[counterclockwise]{rotating}   % Landscape page setup for large tables
\usepackage{multirow}		% Table styling
\usepackage{tabularx}		% Control Column width
\usepackage[flushleft]{threeparttable}	% Allows for three part tables with a specified notes section
\usepackage{threeparttablex}            % Lets threeparttable work with longtable

% Create new environments so endfloat can handle them
% \newenvironment{ltable}
%   {\begin{landscape}\centering\begin{threeparttable}}
%   {\end{threeparttable}\end{landscape}}
\newenvironment{lltable}{\begin{landscape}\centering\begin{ThreePartTable}}{\end{ThreePartTable}\end{landscape}}

% Enables adjusting longtable caption width to table width
% Solution found at http://golatex.de/longtable-mit-caption-so-breit-wie-die-tabelle-t15767.html
\makeatletter
\newcommand\LastLTentrywidth{1em}
\newlength\longtablewidth
\setlength{\longtablewidth}{1in}
\newcommand{\getlongtablewidth}{\begingroup \ifcsname LT@\roman{LT@tables}\endcsname \global\longtablewidth=0pt \renewcommand{\LT@entry}[2]{\global\advance\longtablewidth by ##2\relax\gdef\LastLTentrywidth{##2}}\@nameuse{LT@\roman{LT@tables}} \fi \endgroup}

% \setlength{\parindent}{0.5in}
% \setlength{\parskip}{0pt plus 0pt minus 0pt}

% Overwrite redefinition of paragraph and subparagraph by the default LaTeX template
% See https://github.com/crsh/papaja/issues/292
\makeatletter
\renewcommand{\paragraph}{\@startsection{paragraph}{4}{\parindent}%
  {0\baselineskip \@plus 0.2ex \@minus 0.2ex}%
  {-1em}%
  {\normalfont\normalsize\bfseries\itshape\typesectitle}}

\renewcommand{\subparagraph}[1]{\@startsection{subparagraph}{5}{1em}%
  {0\baselineskip \@plus 0.2ex \@minus 0.2ex}%
  {-\z@\relax}%
  {\normalfont\normalsize\itshape\hspace{\parindent}{#1}\textit{\addperi}}{\relax}}
\makeatother

\makeatletter
\usepackage{etoolbox}
\patchcmd{\maketitle}
  {\section{\normalfont\normalsize\abstractname}}
  {\section*{\normalfont\normalsize\abstractname}}
  {}{\typeout{Failed to patch abstract.}}
\patchcmd{\maketitle}
  {\section{\protect\normalfont{\@title}}}
  {\section*{\protect\normalfont{\@title}}}
  {}{\typeout{Failed to patch title.}}
\makeatother

\usepackage{xpatch}
\makeatletter
\xapptocmd\appendix
  {\xapptocmd\section
    {\addcontentsline{toc}{section}{\appendixname\ifoneappendix\else~\theappendix\fi: #1}}
    {}{\InnerPatchFailed}%
  }
{}{\PatchFailed}
\makeatother
\keywords{keywords\newline\indent Word count: X}
\usepackage{lineno}

\linenumbers
\usepackage{csquotes}
\usepackage{bookmark}
\IfFileExists{xurl.sty}{\usepackage{xurl}}{} % add URL line breaks if available
\urlstyle{same}
\hypersetup{
  pdftitle={Changes in Locus of Control Orientation Across College Major Pathways},
  pdfauthor={Varsha Venkatesh1 \& Moin Syed1},
  pdflang={en-EN},
  pdfkeywords={keywords},
  hidelinks,
  pdfcreator={LaTeX via pandoc}}

\title{Changes in Locus of Control Orientation Across College Major Pathways}
\author{Varsha Venkatesh\textsuperscript{1} \& Moin Syed\textsuperscript{1}}
\date{}


\shorttitle{Locus of Control and Major Pathways}

\authornote{

Add complete departmental affiliations for each author here. Each new line herein must be indented, like this line.

Enter author note here.

The authors made the following contributions. Varsha Venkatesh: Conceptualization, Writing - Original Draft Preparation, Writing - Review \& Editing; Moin Syed: Writing - Review \& Editing, Supervision.

Correspondence concerning this article should be addressed to Varsha Venkatesh, 75 E River Pkwy, Minneapolis, MN 55455. E-mail: \href{mailto:venka288@umn.edu}{\nolinkurl{venka288@umn.edu}}

}

\affiliation{\vspace{0.5cm}\textsuperscript{1} University of Minnesota}

\abstract{%
One or two sentences providing a \textbf{basic introduction} to the field, comprehensible to a scientist in any discipline.
Two to three sentences of \textbf{more detailed background}, comprehensible to scientists in related disciplines.
One sentence clearly stating the \textbf{general problem} being addressed by this particular study.
One sentence summarizing the main result (with the words ``\textbf{here we show}'' or their equivalent).
Two or three sentences explaining what the \textbf{main result} reveals in direct comparison to what was thought to be the case previously, or how the main result adds to previous knowledge.
One or two sentences to put the results into a more \textbf{general context}.
Two or three sentences to provide a \textbf{broader perspective}, readily comprehensible to a scientist in any discipline.
}



\begin{document}
\maketitle

\section{Methods}\label{methods}

We report how we determined our sample size, all data exclusions (if any), all manipulations, and all measures in the study.

\subsection{Participants}\label{participants}

Participants were 96 undergraduate students enrolled at a large, public university in the United States. They ranged in age from 18 to 22 years old (\emph{M} = 18.91, \emph{SD} = 1.15). The majority of participants identified as Asian/Pacific Islander (34.38\%) and male (55.21\%). Additionally, the majority (96.88\%) of participants were born in the United States. A detailed summary of participant demographics is available in Table 1.

\begin{table}[tbp]

\begin{center}
\begin{threeparttable}

\caption{\label{tab:descriptive table}Sociodemographic Characteristics of Participants}

\begin{tabular}{lll}
\toprule
Characteristic & \multicolumn{1}{c}{n} & \multicolumn{1}{c}{\%}\\
\midrule
Gender &  & \\
\ \ \ \  Male & 53 & 55.21\\
\ \ \ \  Female & 40 & 41.67\\
\ \ \ \  Nonbinary & 3 & 3.12\\
Race &  & \\
\ \ \ \  Asian/Pacific Islander & 33 & 34.38\\
\ \ \ \  Black/African American & 24 & 25\\
\ \ \ \  Hispanic/Latine & 16 & 16.67\\
\ \ \ \  Multiracial & 12 & 12.5\\
\ \ \ \  Native American & 7 & 7.29\\
\ \ \ \  Other & 4 & 4.17\\
Birthplace &  & \\
\ \ \ \  Born in the US & 93 & 96.88\\
\ \ \ \  Not Born in the US & 3 & 3.12\\
\bottomrule
\addlinespace
\end{tabular}

\begin{tablenotes}[para]
\normalsize{\textit{Note.} N = 96. Participants were 18.91 years old on average (SD = 1.15)}
\end{tablenotes}

\end{threeparttable}
\end{center}

\end{table}

Participants were recruited through a multicultural orientation event, with the first wave of data collection occurring during their first semester of college and the final wave occurring during the second semester of their sophomore year. Participants who participated in at least three waves of the survey were included in analyses.

\subsection{Materials}\label{materials}

\subsubsection{College Major Pathway}\label{college-major-pathway}

College major pathways were coded based on participant self-reported narratives about their path to choosing a college major. Students were assigned one of five pathways: Uncertain (uncertain of their major from T1 to T4), Discovery (undecided at T1, chose major by T4), Redirect (changed major between T1 and T4), Solidification (unsure about initial major at T1, certain about same major at T4), and Certain (certain about same major from T1 to T4).

\subsubsection{Locus of Control}\label{locus-of-control}

Locus of control was measured through the Internal Locus of Control subscale of the 20-item version of the Multi-Measure Agentic Personality Scale (MAPS20) (Côté, Mizokami, Roberts, \& Nakama, 2016). Participants responded to five items on a 5-point Likert scale ranging from 1 (Strongly Disagree) to 5 (Strongly Agree), with higher scores indicating higher internal locus of control.

\subsection{Procedure}\label{procedure}

\subsection{Data analysis}\label{data-analysis}

We used R (Version 4.5.1; R Core Team, 2025) and the R-packages \emph{DescTools} (Version 0.99.60; Signorell, 2025), \emph{dplyr} (Version 1.1.4; Wickham, François, Henry, Müller, \& Vaughan, 2023), \emph{papaja} (Version 0.1.4; Aust \& Barth, 2025), \emph{summarytools} (Version 1.1.4; Comtois, 2025), and \emph{tinylabels} (Version 0.2.5; Barth, 2025) for all our analyses.

\section{Results}\label{results}

\section{Discussion}\label{discussion}

\newpage

\section{References}\label{references}

\phantomsection\label{refs}
\begin{CSLReferences}{1}{0}
\bibitem[\citeproctext]{ref-R-papaja}
Aust, F., \& Barth, M. (2025). \emph{{papaja}: {Prepare} reproducible {APA} journal articles with {R Markdown}}. \url{https://doi.org/10.32614/CRAN.package.papaja}

\bibitem[\citeproctext]{ref-R-tinylabels}
Barth, M. (2025). \emph{{tinylabels}: Lightweight variable labels}. \url{https://doi.org/10.32614/CRAN.package.tinylabels}

\bibitem[\citeproctext]{ref-R-summarytools}
Comtois, D. (2025). \emph{Summarytools: Tools to quickly and neatly summarize data}. \url{https://doi.org/10.32614/CRAN.package.summarytools}

\bibitem[\citeproctext]{ref-cuxf4tuxe92016}
Côté, J. E., Mizokami, S., Roberts, S. E., \& Nakama, R. (2016). An examination of the cross-cultural validity of the Identity Capital Model: American and Japanese students compared. \emph{Journal of Adolescence}, \emph{46}(1), 76--85. \url{https://doi.org/10.1016/j.adolescence.2015.11.001}

\bibitem[\citeproctext]{ref-R-base}
R Core Team. (2025). \emph{R: A language and environment for statistical computing}. Vienna, Austria: R Foundation for Statistical Computing. Retrieved from \url{https://www.R-project.org/}

\bibitem[\citeproctext]{ref-R-DescTools}
Signorell, A. (2025). \emph{DescTools: Tools for descriptive statistics}. \url{https://doi.org/10.32614/CRAN.package.DescTools}

\bibitem[\citeproctext]{ref-R-dplyr}
Wickham, H., François, R., Henry, L., Müller, K., \& Vaughan, D. (2023). \emph{Dplyr: A grammar of data manipulation}. \url{https://doi.org/10.32614/CRAN.package.dplyr}

\end{CSLReferences}


\end{document}
